\documentclass[12pt,a4paper]{report}
\usepackage{graphicx}
\usepackage{amsmath}
\usepackage{fancyhdr}
\usepackage{cite}
\usepackage{framed}
\usepackage{a4wide}
\usepackage{float}
\usepackage{epsfig}
\usepackage{longtable}
\usepackage{enumerate}
\usepackage{afterpage}
\usepackage{multirow}
\usepackage{ragged2e}
\usepackage{gensymb}
\usepackage{amsfonts} 
\usepackage[left=3.5cm,top=1.5cm,right=3cm,bottom=4cm]{geometry}
\usepackage{setspace}           
\usepackage{float}
\usepackage{txfonts}
\usepackage{lipsum}

\newcommand{\Usefont}[1]{\fontfamily{#1}\selectfont}

\usepackage{lscape} % for landscape tables
\renewcommand{\baselinestretch}{1.7} 

\usepackage{blindtext}
\usepackage{xpatch}
\usepackage{url}
\usepackage{leqno}
\usepackage{subcaption}

\linespread{1.5}
\usepackage[intoc, english]{nomencl}
\hyphenpenalty=5000
\tolerance=1000
\usepackage[nottoc]{tocbibind}
\bibliographystyle{IEEEtran}
\renewcommand{\bibname}{References}

%*******************************************************************
%                        Header and Footer   
% This is not required in Technical reports submitted to CET ECE department.
% Please leave it commented                       
%*******************************************************************
%\pagestyle{fancy}
%\fancyhead{}
%\header and footer section
%\renewcommand\headrulewidth{0.1pt}
%\fancyhead[L]{\footnotesize \leftmark}
%\fancyhead[R]{\footnotesize \thepage}
%\renewcommand\headrulewidth{0pt}
%\fancyfoot[R]{\small College of Engineering Trivandrum}
%\renewcommand\footrulewidth{0.1pt}
%\fancyfoot[C]{2020 - 2021}
%\fancyfoot[L]{\small Title of the Seminar/Project}
%*******************************************************************


%*********************Figures*****************************
% Save all figures in the folder figures and include them in your 
% report using the command \includegraphics{figure-name}

\graphicspath{{figures/}}

% figure files can be in jpeg,jpg, png or pdf formats
%*******************************************************************


\begin{document}
    
    
%****The entries in this section are to be filled in by the student with appropriate values *************

% These values are used thoroughout the report 
% please fill in the appropriate values

\gdef \title{How to prepare a Technical report using \LaTeX } % Seminar title
\gdef \author{Student Name}  %student name
\gdef \dept{Electronics and Communication Engineering} %Department
\gdef \degree{Bachelor of Technology} %degree
\gdef \branch{Electronics and Communication Engineering} %branch
\gdef \college{College of Engineering}
\gdef \collegeplace{Trivandrum}
\gdef \rollno{TVE17EC0XY} %KTU Reg No
\gdef \deptabbr{Dept.of ECE} %Dept name abbreviation

\gdef \guide{Prof. Seminar guide} %Seminar guide
\gdef \guidedes{Assistant Professor}%Seminar guide designation

\gdef \semcordinatorA{Prof. Joaquim Ignatious M.}% Seminar coordinator 1 
\gdef \semcordinatorAdes{Assistant Professor}% Seminar coordinator 1 designation

\gdef \semcordinatorB{Prof. Seminar coordinator 2} % Seminar coordinator 2 
\gdef \semcordinatorBdes{Assistant Professor}% Seminar coordinator 2 designation

\gdef \hod{Dr. Santhosh Kumar S.} %Head of Department
\gdef \hoddes{Professor and Head} %HOD designation

\gdef \acadyear{2020 - 21} % Academic year
\gdef \month{November 2020} %Month of Report submission
\gdef \date{21-11-2020} %Date of signing the declaration

%*******************************************************************
% The font pages. The source tex files are there in the folder
\include{coverpage} %Unless essential Do not edit this tex file



%%********************Certificate*******************

% To print name of only the seminar coordinator 1 in the certificate page
\include{certificate1} 

% To print names of both the seminar coordinators in the certificate page
%\include{certificate2} %Please uncomment this and comment the previous line

%%***************************************************


\include{declaration} %Unless essential Do not edit this tex file

\pagenumbering{roman} 

%%********************************Abstract***********************
\include{abstract} % Please type in the abstract in this tex file abstract.tex

%%***************************************************
% Default Acknowledgement page
\include{acknowledgement}  %Unless essential Do not edit this tex file


%%***************************************************
%%**If you have only one seminar coordinator faculty member
% please comment the above line and uncomment this line

%\include{acknowledgement1}  %Unless essential Do not edit this tex file
%*******************************************************************

\thispagestyle{empty}
\newpage
    
%%**********************Table of Contents***********************
\tableofcontents
\listoffigures
\listoftables
\include{symbol} %List of Symbols (Optional) comment if not required.
% symbold may be added in the file symbol.tex

%%********************Body of the report**********
% Arabic numbering is used in the body of the report

\cleardoublepage
\setcounter{page}{1}
\pagenumbering{arabic}

%%********************Chapter 1**********
\chapter{Introduction}
\lipsum[1] % Please comment this line and type in the introduction chapter

%%********************Chapter 2**********
\chapter{Literature Review}
Technical writing is writing or drafting technical communication used in technical and occupational fields\cite{india}, such as computer hardware and software\cite{rpi}, engineering, chemistry, aeronautics, robotics, finance\cite{japan}, medical, consumer electronics, biotechnology, and forestry. Technical writing encompasses the largest sub-field in technical communication. See figure \ref{net2} that shows the autonomous systems in Internet.

\begin{figure}[h!]
    \centering
    \includegraphics[width=0.9\linewidth]{ospf}
    \caption{Autonomous System Hierarchy}
    \label{net2}
\end{figure}

\section{section1}
\lipsum[2] % Please comment this line and type in the introduction chapter


\subsection{title 2}
\lipsum[3] % Please comment this line and type in the introduction chapter

\noindent The system is described by the equation \ref{sys_eq1} below. Here y is the ordinate and x is the abscissa , m is the slope and c a constant.

\begin{equation} \label{sys_eq1}
y = mx + c
\end{equation}
\noindent Page centered and unnumbered multiple equations. The * symbol supresses equation numbering.
% Page centered and unnumbered equations
\begin{align*}
2x - 5y &=  8 \\ 
3x + 9y &=  -12
\end{align*}

\noindent Side by side figures can be created using this environment. See fig \ref{wave} below.
\begin{figure}[h!]
    \centering
    \begin{subfigure}[b]{0.4\textwidth}
        \includegraphics[width=\textwidth]{sinewave}
        \caption{Sine Wave}
        \label{fig:1}
    \end{subfigure}
    \hspace{20mm}
    \begin{subfigure}[b]{0.4\textwidth}
        \includegraphics[width=\linewidth]{cosine}
        \caption{Cosine Wave}
        \label{fig:2}
    \end{subfigure}
\caption{The Sine and Cosine waves}
\label{wave}
\end{figure}

%%********************Chapter 3**********
\chapter{Results}
\lipsum[5-7] % Please comment this line and type in the results chapter
    
\begin{table}[h!]
    \centering
    \caption{test table}
    \vspace*{5pt}
    \begin{tabular}{|c|c|c|}
        \hline
        Sl. No & Item 1 & Itm 2 \\ \hline
        1      & 37     & 45    \\ \hline
        2      & 42     & 23    \\ \hline
        3      & 47     & 1     \\ \hline
        4      & 52     & -21   \\ \hline
        5      & 57     & -43   \\ \hline
        6      & 62     & -65   \\ \hline
        7      & 67     & -87   \\ \hline
        8      & 72     & -109  \\ \hline
        9      & 77     & -131  \\ \hline
        10     & 82     & -153  \\ \hline
    \end{tabular}
\end{table}

%%********************Chapter 4**********
\chapter{Conclusion}
\lipsum[2] 

%%********************References**********
%%****This template uses IEEE bibliography style

 \begin{thebibliography}{99}
    \bibitem{india} HU, Yun Chao, et al., \emph{Mobile edge computing?A key technology
        towards 5G}, ETSI white paper, 2015, vol. 11, no 11, p. 1-16.
    
    
    \bibitem{rpi}
    @online{ Raspberry pi,
        \url{https://www.raspberrypi.org/}
        Online; accessed 10-June-2019
    }
    
    \bibitem{japan} HU, Yun Chao, et al., \emph{Mobile edge computing?A key technology
        towards 5G}, ETSI white paper, 2015, vol. 11, no 11, p. 1-16.       
\end{thebibliography}

\end{document}